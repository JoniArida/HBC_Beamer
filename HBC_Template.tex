\documentclass{beamer}
\usetheme{HBC}

\usepackage[utf8]{inputenc}
\usepackage{ngerman}
\usepackage[T1]{fontenc}
\usepackage{lmodern}
\usepackage[expert]{mathdesign}
\usepackage{mathrsfs}
\usepackage{tikz}


\title{\textbf{Präsentationen mit dem HBC-Theme}}
\subtitle{Kurze Anleitung zur Benutzung des Layouts}
\author{J. D.}
\date{29.04.2014}
\institute{Hochschule Biberach\\Studiengang Gebäudeklimatik}
\hbclogo{GFX/HBC_logo.pdf}

\begin{document}
\titlepage

\begin{frame}
\frametitle{Inhalt}
\tableofcontents
\end{frame}

%=============================================================================================

\section{Benutzen des Themas}
\sectionframe
\begin{frame}
\frametitle{Installation}
Um das Thema muss es heruntergeladen und am besten in den \LaTeX-Ordner des Benutzers kopieren. (~/texmf/tex/latex/)\linebreak
Es kann aber auch in dem Ordner abgelegt werden in dem Projektordner befinden in dem das Dokument erstellt wird.
\end{frame}

\begin{frame}
\frametitle{Mathe}
Hier etwas Mathe:
\begin{equation}
F(s):=\int\limits_{0}^{\infty}f(t)e^{-st}\mathrm{d}t=\mathscr{L}\{f(t)\}
\end{equation}
\end{frame}

\begin{frame}
\frametitle{Animations}
Animationen
\pause
können auch Genutzt werden:
\begin{itemize}[<+->]
  \item Das steht von Anfang an da!
  \item Erscheint bei ersten Klick
  \item Das beim zweiten
  \item usw…
  \item Am besten mal die Beamer Doku lesen!
\end{itemize}
\end{frame}

\subsection{Subsection}
\subsectionframe
\begin{frame}
\frametitle{Subsections}
Es lassen sich auch Subsection erstellen.\linebreak
Die Inhaltsverzeichnisse sind optional.
\end{frame}

\begin{frame}
Die  Seiten brauchen alle einen Titel.\linebreak
Sonst sehen die Seiten so aus…
\end{frame}

\subsection{Blocks}
\subsectionframe
\begin{frame}
\frametitle{Blocks}
\begin{definition}[Block]
In Beamer kann man auch Blocks nutzen. Dies ist ein Definition Block.
\end{definition}

\begin{theorem}[Theorem Block]
$a^n + b^n = c^n, n \leq 2$ 
\end{theorem}

\begin{alertblock}{Ein Alert Block}
Alarm…!?
\end{alertblock}
\end{frame}


\section{ToDo} % (fold)
\label{sec:Was moch zu tun ist}
\sectionframe

\begin{frame}

\frametitle{Notation}
\begin{definition}[ToDo]
\begin{enumerate}[<+->]
\item Es sollten die Farben der Blocks noch an das „corporate Design“ angepasst werden.
\item Es fehlt noch eine Folie zum Ende. Sowas wie: „Vielen Dank für die Aufmerksamkeit.“
\end{enumerate}


\end{definition}
\end{frame}


\end{document}